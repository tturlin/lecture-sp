\documentclass[a4paper]{rapport}

\graphicspath{{Pictures/}}

\begin{document}
\part{Quelques notion d'astrophysique}
On appel \textit{Cycle de la matière galactique} les différentes étapes et états par lesquels transite la matière telle que nous la connaissons. C'est par ce cycle que notre univers s’enrichit en éléments lourd et que les étoiles ainsi que tous les objets visible sont créé, il va être question dans cette première partie de décrire ces différentes étapes et de mieux comprendre comment évolue la matière au sein de notre univers.
\nl
Nous allons prendre comme point de départ de notre cycle le \textbf{milieu interstellaire}, il à une masse d'environ $5.10^{9}M_{\odot}$ et est constitué aux trois quart d'hydrogène, d'hélium à niveau d'un quart et que poussière pour ce qui est des dixième de pourcent restant (il constitue le réservoir de matière de notre univers). On considère que dans ce milieu la densité est environ d'une particule par $cm^3$, bien que cela soit extrêmement faible, certaines régions sont plus densément peuplée que d'autres a cela on rajoute l'hydrodynamique, l'auto gravité... on observe sur des périodes de temps assez faibles (quelques millions d'année) l'apparition de nuage moléculaire.\\ Ces nuages forment le milieu interstellaire \textbf{dense}, sa masse reste de l'ordre de $10^{9}M_{odot}$ mais sa composition à évolué pour être constitué à environ $75\%$ de dihydrogène (d’où le nom de nuage moléculaire). C'est à partir de ce moment que les choses vont s'emballer, les zones les plus denses des nuages vont attirer plus de matière à eux entraînant une augmentation de la densité etc... jusqu’à ce que l'instabilité gravitationnelle soit trop forte et que le nuage s’effondre sur lui même.\\
Il résulte de cet effondrement, la formation d'une \textbf{Proto-étoile}. Durant cette phase qui va durer quelques centaines de milliers d'années, la matière va continuer à tourner s'agglomérer autour du centre de gravité. Au fur et à mesurer de ce processus la proto-étoile va voir sa température augmenter, les particules du gaz vont s'échauffer continuellement jusqu’à atteindre une température critique qui permettra "l'allumage" de la réaction de fusion de l'hydrogène (plusieurs million de Kelvins), une \textbf{Étoile} est née.\\
Le cycle de vie d'une étoile est assez complexe, pour simplifier nous dirons que l'étoile fusionne de l’hydrogène en son sein et que cette réaction et à l'origine de la pression interne de l'étoile, pression qui vient compenser la force de gravité qui tend à faire s’effondrer l'étoile sur elle-même. Une fois que l'étoile à épuiser tout son hydrogène, la gravité va reprendre le dessus faire s’effondrer l'étoile, augmenter la pression et la température, réunissant les conditionsg pour fusionner l'hélium, stoppant effondrement etc...\footnote{Ce processus s’arrête au fer car aucune étoile ne peut fusionner le fer, en effet la fusion post fer consomme de l'énergie la ou la fusion pré fer en produisait.}.\\
Le cycle de vie d'une étoile est représenté par un diagramme HR (\textit{Hertzsprung-Russel}) qui permet de voir l'évolution d'une étoile le long de la séquence principale (où elle passera le plus clair de son existence) mais également de savoir quel objet astronomique il résultera de sa mort.\\
Une fois que l'étoile aura épuisé tout son carburant elle "mourra" sois en passant par la \textbf{Nébuleuse planétaire} sois en \textbf{Supernovæ} si l'étoile était suffisamment massive.\\
Ces deux phénomènes permettent de rendre au milieu interstellaire un grand nombre de nouvelle particules, enrichissant de manière considérable celui-ci et permettant l'apparition future d'objet céleste plus complexe en terme de composition.
\nl
Nous rentrerons plus en détails sur les phénomènes de Nébuleuse planétaire et de Supernovæ dans les parties consacrées aux \textbf{Étoiles à neutron} et au \textbf{Naines blanches}, car nous verrons que dans la plupart des cas \footnote{La réalité astrophysique est bien plus complexe que cela et la manière de former certains objet de notre univers peut être assez exotique.} Ces objets sont le résultats des Nébuleuses et des Supernovæ.

\part{Les naines blanches}
\section{Nébuleuse planétaire}
Comme dit précédemment nous allons revenir sur le phénomène de Nébuleuse planétaire qui donne naissance à la plupart des naines blanches de l'Univers.\nl
Pour commencer il faut savoir qu'une manière de classer les étoiles consiste à les ordonnée selon leurs masse, et donc on estime que seuls les étoiles de masse moyenne (C.à.D de moins de $10M_{\odot}$) donnent naissance à des Naines blanches. Le processus qui mène à la création d'une telle étoile est le suivant.\nl
Tout d'abord comme nous l'avons vu dans notre introduction, une étoile va fusionner son hydrogène en hélium \footnote{On a généralement deux moyen d'y parvenir, sois par cycle proton-proton qui va par le biais du deutérium et du tritium/hélium 3 va donner de l'hélium 4, sois par le cycle CN0}\\
A partir de ce moment, la gravitation redevient alors la force dominante et l'étoile s’effondre sur elle même, les nouvelles conditions de pression et de température permette la fusion de l’hélium \footnote{Par un procédé appelé réaction triple alpha lorsque que la température aura atteint les $10^8 K$} cette fusion produit principalement du carbone et de l'oxygène et dégage une une importante quantité d'énergie qui ne se contente plus de contrebalancer l'effondrement gravitationnel mais qui fais gonfler l'étoile de manière impressionnante \footnote{On considère que lorsque le soleil entrera dans sa phase de géante rouge, sa taille aura tellement augmenter qu'il atteindra l'orbite de la Terre.} la transformant ainsi en géante rouge.\\
Les réserves d'hélium s’épuisant relativement vite, l’effondrement de l'étoile reprend assez vite, malheureusement pour la plupart des étoiles cet effondrement n'engendre pas une pression et une température suffisante pour amorcer le processus de fusion du carbone laissant l'étoile se réduire à un noyau solide de carbone et d'oxygène. Les couches externe de l'étoiles vont alors "rebondir" sur ce noyau. En réalité, le phénomène est plus complexe que ça et c'est par la pression de radiation que ces couches vont se faire souffler lors de la violente contraction du cœur de l'étoile.\\
Ce nuage de particule va donc lentement (à l'échelle de l'univers, en réalité ce nuage se déplace 100 000 $km/h$) se faire expulser allant ainsi alimenter le milieu interstellaire en élément plus riche, c'est la nébuleuse planétaire. Il ne reste plus alors que le noyau au centre de cet immense nuage et c'est ce "noyau" qui constitue la naine blanche.

\section{Quelques caractéristiques des naines blanches}
Une des première naine blanche à avoir été observé est Sirius B compagnon de l'étoile Sirius de la constellation du grand chien, ont avait remarqué en 1844 des anomalie dans la manière dont se déplaçait Sirius imaginé que ces anomalie pouvait être du à un compagnon encore invisible jusque la. Quelque année plus tard en 1862 ont réussi à observé Sirius B et a se rendre compte que cette étoile est bien une naine blanche.
\nl

Bien que leurs tailles et leurs masse sois relativement variée, pour ce cours nous ferons l'approximation (partagée par la communauté scientifique) qu'une naine blanche
est un objet de la masse du soleil contenu dans un volume semblable à la Terre,
$$R\simeq 5000km\qquad M\simeq 10^{30}kg\qquad \rho\simeq 5.10^{6}g/cm^3$$
on note qu'elle figure parmi les objet les plus denses de l'univers.
La température au sein de ces monstres est de l'ordre de $10^7 K$ et le cœur de la naine, comme nous l'avons dit un peu plus haut, n'est plus constitué que des reste des réactions nucléaire, c'est à dire principalement de l'oxygène et du carbone, plus précisément d'un plasma constitué de noyau d'oxygène et d'électron.

\section{Théorie des naines blanches}
Nous avons vus que les étoiles de la séquence principales sont en équilibre car les réactions nucléaires qui se produisent dans leurs cœurs permettent de contrebalancer l'effondrement gravitationnel, cependant qu'en est-il des naines blanches? pendant longtemps les scientifiques se sont posé la question étant donnée que nous savons les naines blanches exemptent de toutes réactions nucléaires en plus d'avoir à subir d'incroyable force de pression gravitationnelle.\\
Un élément de réponse à été apporter en 1925 par le britannique Ralph Fowler, qui propose d'appliquer les principes de la mécaniques quantiques au gaz d'électrons présent dans le noyau des naines blanches. Ce que propose Fowler c'est que les électron représente en réalité un gaz quantique dégénéré et que ce gaz induit une pression quantique qui vient assurer la stabilité de la naine blanche.\\
Le but de la partie d'aujourd'hui va être de retrouver ces résultats et de redémontrer qu'effectivement la stabilité d'une naine blanche provient d'une pression quantique induites par un gaz d'électrons dégénéré relativistes.
\subsection{Gaz dégénéré...}

\begin{itemize}
  \item Pour commencer l'énergie potentielle gravitationnelle d'un corps de masse M et de dimension R vaut $E_g = -G \frac{M^2}{R}$
  \item Pour se faire une idée de l’intensité de ces force on imagine un accroissement DR de l'objet $dE_g = G \frac{M^2}{R^2}dR$
  \item l'augmentation d'énergie correspond à un travail $\delta W = -dE_g$ avec $dW  =P_gdV = P_g 4\pi R^2 dR$
  \item on obtient au final $P_g\simeq -\frac{G}{4_pi}\frac{M^2}{R^4} = -\frac{6.67.10^{-11}}{4\pi}\frac{M^2}{R^4} = \boxed{-10^{33} Pa}$
\end{itemize}
\nl
\begin{itemize}
  \item Naine blanche rayonne $\rightarrow$ perte d'énergie et donc refroidissement
  \item Refroidissement entraîne théoriquement une recombinaisons noyaux/électrons
  \item Calcul du nombre de noyaux par unité de volume
\end{itemize}

$$\frac{N}{V}\simeq \frac{\rho}{A}*N_A \simeq 2.10^{29}cm^{-3}$$
\begin{itemize}
  \item En inversant on à le volume dispose disponible pour chaque noyaux $5.10^{-6}A$
  \item Volume occupé par un atome? 1 million de fois plus grand
  \item Aucun phénomène physique connu capable de contrebalancer les forces gravitationnelle pour augmenter à ce point le volume.
\end{itemize}
\nl
On s’intéresse à la température de Fermi (température de dégénérescence) du gaz d'électrons
\begin{itemize}
  \item Même manière de calculer que pendant le cours de physique des matériaux
  $$T_F = \frac{\hbar^2}{2m_e K}\left(3\pi^2 \frac{N_e}{V}\right)^{\frac{2}{3}}$$
  \item Température de Fermi $\boxed{5.10^9 K}$
  \item Avec $\frac{N_e}{V}\simeq \frac{\rho}{A}*Z*N_A = 1.5.10^{30}cm^{-3}$
  \item Température du gaz d'électron bien en deçà de la température de Fermi $\rightarrow$ Gaz d'électron totalement dégénérée.
\end{itemize}
\nl

\begin{itemize}
  \item Pression d'un gaz de fermions dégénérée $$\frac{2}{15\pi^2\hbar^3}(2m_e)^{\frac{3}{2}}(kT_f)^{\frac{5}{2}}$$
  $\hbar = 1.054.10^{-34}$
  $9.1.10^{-31}$
  \item Pression du gaz d'électrons $\simeq \boxed{10^{22} Pa}$
  \item On voit qu'un gaz d'électron dégénérée ne suffit pas à expliquer la stabilité de la naine blanche, la pression n'est pas assez forte.
\end{itemize}

\subsection{... d'électrons relativistes}
\begin{itemize}
  \item Toujours par rapport au cours de matériaux on à les énergie mises en jeu qui valent $E = k*T_F \simeq \boxed{10^5 eV}$
  \item Les électrons sont relativistes : $E = \sqrt{p^2c^2 + m^2c^4}$
  \item On en profite pour calculer l'impulsion de Fermi $ \sqrt{\frac{E^2 - m^2c^4}{c^2}} = 3.3.10^{-4}$
  \item travailler avec des formules relativistes de gaz dégénérée afin de trouver la condition d'équilibre de la naine blanche.
\end{itemize}
\nl
\begin{itemize}
  \item On peut montrer que la pression d'un gaz relativiste est donnée par $P = \frac{J}{V}$ avec J le grand potentiel (potentiel de la distribution micro canonique).
  \item Les formules sont assez complexe et assommantes, mais pour des gaz ou le gaz est complètement dégénérée (température très faible devant la température de Fermi), le grand potentiel s'écrit :
  $$J = - \frac{V}{3\pi^2\hbar^3}\int_0^{p_F} p^3 \frac{dE}{dp} dp$$
  \item La pression des particules relativiste s'écrit donc $$P = \frac{c^2}{3\pi^2\hbar^3}\int_0^{p_F} \frac{p^4}{\sqrt{p^2c^2 + m^2c^4}} dp $$
  \item on pose le chgt de variable $x = \frac{p}{mc}$
  \begin{eqnarray*}
    P &=& \frac{c^2}{3\pi^2\hbar^3}\int_0^{x_F} \frac{x^4 m^4 c^4}{\sqrt{x^2 m^2 c^4 + m^2c^4}}mcdx\\
    &=&\frac{c^2}{3\pi^2\hbar^3}\int_0^{x_F}\frac{x^4m^5c^5}{\sqrt{m^2c^4(x^2 + 1)}}dx\\
    &=&\frac{m^4c^5}{3\pi^2\hbar^3}\int_0^{x_F} \frac{x^4}{\sqrt{x^2 + 1}}\\
    &=& \boxed{\frac{c}{12\pi^2 \hbar^3}p_F^4\left[1 - \frac{m^2 c^2}{p_F^2}\right]}
  \end{eqnarray*}
  \item On calcule et on obtient une pression pour le gaz dégénérée d'électron relativiste de $\boxed{10^{36}Pa}$

\end{itemize}
Au arrondi près et au erreurs de calculs, on trouve une pression bien supérieur à celle du seuls gaz d’électrons dégénérée et ont voit bien que cette pression est suffisante pour permettre de contrebalancer la pression gravitationnelle et maintenir la naine blanche en équilibre.








\end{document}
